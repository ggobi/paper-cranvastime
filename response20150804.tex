%% LyX 2.0.6 created this file.  For more info, see http://www.lyx.org/.
%% Do not edit unless you really know what you are doing.
\documentclass[english]{article}
\usepackage[T1]{fontenc}
\usepackage[latin9]{inputenc}
\usepackage{geometry}
\geometry{verbose,tmargin=3.5cm,bmargin=4cm,lmargin=3.3cm,rmargin=3.5cm}
\usepackage{babel}
\usepackage{hyperref}
\pagestyle{empty}
\setlength{\parskip}{\bigskipamount}
\setlength{\parindent}{0pt}
\usepackage{graphicx}
\usepackage{setspace}
\usepackage{tgpagella}
\onehalfspacing

\begin{document}
\begin{flushright}
August 3, 2015
\end{flushright}
Dear Editor,

\vspace{.5cm}
Thanks a lot for the review on our manuscript JCGS-14-243.
We have made the revision based on your comments. Below is
the point-by-point response.

\begin{itemize}

\item Length of the manuscript. We have tried our best to shorten
the paper to 28 pages, by moving some secondary information to
the appendix.

\item Comparison to \url{http://survey.timeviz.net/}. We visited
the website and found it very nicely categorized time visualization.
Comparing to their idea, our paper focused on the taxonomy of
interactions. We referenced Aigner's (the author of the website)
paper in Section 1.2.

\item Figure 11. After shortening, Figure 11 is now Figure 10. We
added a note that it was an intermediate position of faceting.
About the "visual notation indicating how brushing conflicts are
prevented", our response is that the figure would be refered before
the discussion of the linking issue, since we had to shorten the
pages. So logically the notation is not necessary. In addition,
the corresponding data do not have multiple variables as a
"wide data". Though the linking conflict still exists between
the time plot and the map (i.e., between the data and geographical
information), adding the visual notation may not clarify the
problem but would make it more complicated.

\item Axis rescaling issues. Axis rescaling is a basic interaction
named "zooming" in our work. Zooming is not of our interest because
it does not change the relative distance between points, i.e.,
zooming does not transform the data as the other interactions
like wrapping or faceting. We added an explanation in the first
paragraph of Section 4.

\item Example of polar coordinates. To get an interactive example
of polar coordinates requires much coding that is hard to be
completed in two months. We would like to leave this as the future
work.

\end{itemize}

Again, thank you for your feedback to our work. Please let us know
if any further revision is needed.

\bigskip{}

Sincerely,

Xiaoyue Cheng

\end{document}
